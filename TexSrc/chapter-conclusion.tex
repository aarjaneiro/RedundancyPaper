%! Author = ajst
%! Date = 2021-02-28

% Preamble
\chapter{Concluding Remarks}\label{ch:discussion}

In all, with respect to the hypothesized behaviour specified in Conjecture~\ref{conj}, we were able to find support for asymptotic independence between queues in
$\mathbf{X}^{(N)}_{t}$ as $t\rightarrow \infty$ and $N\rightarrow \infty$, respectively at $r \in {1,2}, d \equiv 2$.
Moreover, after iterating the limits, we found support for the system turning deterministic in law.
Along with the model developed for the system as outlined in Chapter~\ref{ch:model-specification}, a framework for formally proving the result in a manner similar to~\cite{bramson_asymptotic_2012}
is both introduced and demonstrated to likely be true for the tested parameters.

Outside of demonstrating the conjectured behaviour, this research has also highlighted some potentially-interesting directions
for future research into stochastic simulation and independence testing. 
With respect to simulations, \textit{ParallelQueue} was of course developed primarily for the systems studied in this paper.
While it was demonstrably fast in terms of single-process simulation, parallelizing in order to efficiently simulate multiple replications of the process
at once proved to be a memory-intensive task and necessitated porting the codebase into Cython.
This highlights the need for the development of a low-level DES framework which can efficiently manage memory and resources between simultaneous simulations of a process.

In terms of independence testing, this research has demonstrated the need for further exploration into methods for discrete processes and group independence.
Clearly for a queueing process which does not ``explode'', we would expect that queue counts do not increase indefinitely~\cite{gardner_redundancy-d_2017}.
In our tests, such seems to have left queues with relatively little variance to study (in most cases, queue counts were binary for $N>5$), perhaps biasing results against the null.
Lastly, with respect to group dependence, the fact that a counting process would have only been able to depend on a maximum of $d-1$ others could perhaps have been utilized when determining a statistic for group independence
across time samples.
Future research in independence testing, especially with respect to $d$HSIC/HSIC might therefore benefit from exploring bootstrap estimators for constrained groupings.

